\chapter{Basics of Category Theory}

A category $\Cscr$ consists of the following data:
\begin{enumerate}
  \item A collection of objects, denoted $\ob\Cscr$.
  \item For every pair of objects $X,Y$, a collection of arrows $\Hom(X,Y)$.
  \item A composition rule
        \[ (f,g) \mapsto f \circ g : \Hom(Y,Z) \times \Hom(X,Y) \to \Hom(X,Z). \]
  \item An identity moprhism $\one_{X} \in \Hom(X,X)$ for every object $X$.
\end{enumerate}
This data must satisfy the following axioms:
\begin{enumerate}
  \item Composition of morphisms is associative: $f \circ (g \circ h) = (f \circ g) \circ h$.
  \item The identity morphism $\one_{X}$ is a two-sided identity for composition of morphisms.
\end{enumerate}
We will usually abuse the notation and write $X \in \Cscr$ to mean that $X$ is an object of $\Cscr$, i.e.~$X$ is a member of $\ob\Cscr$.
If we wish to specify the category in $\Hom(X,Y)$, we will write $\Hom_{\Cscr}(X,Y)$ or $\Hom^{\Cscr}(X,Y)$, although $\Cscr$ will be left implicit whenever possible.
Given $X,Y \in \Cscr$, we may also write $f : X \to Y$ to mean that $f$ is a member of $\Hom(X,Y)$ when $\Cscr$ is understood from context.
Finally, we will often omit the $X$ from $\one_{X}$ and write simply $\one$ when it can be deduced from context.

A morphism $f : X \to Y$ is called an \emph{isomorphism} if it has a two-sided inverse.
Explicitly, this means that there exists a morphism $g : Y \to X$ such that $f \circ g = \one$ and $g \circ f = \one$.

\begin{example}
  The category $\Set$ of sets.
  Its objects are sets, and, given two sets $S,T$, a morphism $f : S \to T$ is simply a function from $S$ to $T$.
\end{example}

\begin{example}
  The category $\Top$ of topological spaces.
  Its objects are topological spaces, and, given two spaces $X$ and $Y$, a morphism $f : X \to Y$ is a continuous map from $X$ to $Y$.
\end{example}

\begin{example}
  The category $\Group$ of groups.
  Its objects are groups, and, given two groups $G$ and $H$, a morphism $f : G \to H$ is a group homomorphism from $G$ to $H$.
\end{example}

%%% Local Variables:
%%% mode: latex
%%% TeX-master: "main"
%%% End:
